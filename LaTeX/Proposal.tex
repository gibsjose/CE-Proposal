\documentclass[12pt]{article}

% Commands
\newcommand{\ASSNMT}{GVSU Computer Engineering Program}
\newcommand{\CLASS}{Review, Critique, and Proposal}
\newcommand{\Footer}{Grand Valley State University}

\newcommand{\DATE}{July 2015}

% Packages
\usepackage[utf8]{inputenc}
\usepackage[T1]{fontenc}
\usepackage{lmodern}
\usepackage{pdflscape}
\usepackage{geometry}
\usepackage[usenames,dvipsnames]{xcolor}
\usepackage{graphicx}
\usepackage{mathtools}
\usepackage[justification=centering]{caption}
\usepackage{amssymb}
\usepackage[pdftex, colorlinks=true, urlcolor=blue, linkcolor=black, pdfborderstyle={/S/U/W 0}]{hyperref} % this disables the boxes around links]
\usepackage{float}
\usepackage{listings}
% \usepackage{color}
\usepackage{enumitem}
\usepackage{fancyhdr}
\usepackage{caption}
\numberwithin{figure}{section}
\usepackage{amsmath}

\numberwithin{equation}{section}
% lstlisting
\definecolor{dkgreen}{rgb}{0,0.6,0}
\definecolor{gray}{rgb}{0.5,0.5,0.5}
\definecolor{mauve}{rgb}{0.58,0,0.82}
\lstset
{
  frame=single,
  frameround=tttt,
  language=C,
  numberstyle=\tiny\color{gray},
  keywordstyle=\color{blue},
  commentstyle=\color{dkgreen},
  stringstyle=\color{mauve},
  tabsize=3,
  breaklines=true,
  basicstyle={\small\ttfamily},
  xleftmargin=\fboxsep,
  xrightmargin=-\fboxsep,
  numbers = left,
  stepnumber = 5,
  firstnumber = 1
}

% macro for appendix to be printed as "Appendix A {name of appendix}"
% instead of "A {name of appendix}"
% From: http://tex.stackexchange.com/questions/160839/having-appendix-a-instead-of-a-appendix
\makeatletter
%% The "\@seccntformat" command is an auxiliary command
%% (see pp. 26f. of 'The LaTeX Companion,' 2nd. ed.)
\def\@seccntformat#1{\@ifundefined{#1@cntformat}%
   {\csname the#1\endcsname\quad}  % default
   {\csname #1@cntformat\endcsname}% enable individual control
}
\let\oldappendix\appendix %% save current definition of \appendix
\renewcommand\appendix{%
    \oldappendix
    \newcommand{\section@cntformat}{\appendixname~\thesection\quad}
}
\makeatother
% Sign and Date command
\newcommand{\namesigdate}[2][5cm]{%
  \begin{tabular}{@{}p{#1}@{}}
    #2 \\[2\normalbaselineskip] \hrule \\[0pt]
    {\small \textit{Signature}} \\[2\normalbaselineskip] \hrule \\[0pt]
    {\small \textit{Date}}
  \end{tabular}
}


\begin{document}
% =====----- Initial Set Up -----=====
% Title Page
\newgeometry{top=2cm,left=1cm,bottom=1cm,right=1cm}
\begin{flushleft}
\pagenumbering{gobble}

\textsc{\LARGE \bfseries \ASSNMT}\\

\textsc{\Large \CLASS}\\[0.2cm]
\linethickness{0.5mm}
{\color{NavyBlue}\line(1,0){350}} \\ [1.0cm]

\begin{flushleft} \large
\begin{tabular}{lll}
Written By: & Joe Gibson    & \href{mailto:gibsjose@mail.gvsu.edu}{gibsjose@mail.gvsu.edu}\\
              &               & \\
Comments and Review By: & Jesse Millwood  &       \href{mailto:millwooj@mail.gvsu.edu}{millwooj@mail.gvsu.edu}\\
        & Kurt VonEhr     & \href{mailto:vonehrk@mail.gvsu.edu}{vonehrk@mail.gvsu.edu}\\
\end{tabular}

\bigskip

\bigskip

% If you don't also put this in a 'tabular' layout, then it fully left-aligns the date text, which makes  it like 0.5cm too far to the left... which really bugs me for some reason
\begin{tabular}{lll}
Date: \DATE
\end{tabular}
\end{flushleft}

\smallskip
{\color{NavyBlue}\line(1,0){350}} \\ [1.0cm]
\section*{Executive Summary} \label{sect:execsum}
This document attempts a review and critique of the Computer Engineering program at Grand Valley State University in the form of a proposal. Both the strengths and weaknesses of the program will be discussed, and additional topics will be proposed.

\bigskip

The goal of this document is to provide feedback from a former student who believes that the program is in a position to greatly enhance an already favorable curriculum. By reviewing this document and taking its contents to heart, the university will be demonstrating its commitment to student feedback and its ability to adapt to changing educational and industrial landscapes.

\bigskip

The following items are the main points of the proposal:

\begin{enumerate}
    \item C++ should be taught to CE students following their introduction to C, rather than teaching them Java, which is never actually used after CIS 163.
    \item Students should design their own ATmega-based development board instead of relying only on the Arduino. They should incorporate circuit protection aspects of the \textit{Ruggeduino}, which they learn about in EGR 326.
    \item Students should be taught version control systems (such as \texttt{git}) to manage code, group projects, and project submissions.
    \item EGR 424 (Design of Microcontroller Applications) and CIS 457 (Data Communications) are both fundamental to Computer Engineering, and should be required courses, rather than electives.
\end{enumerate}

\vfill

% Bottom of the page
\begin{center}
{\large \Footer}
\end{center}
\begin{figure}[H]
  \centering
  \includegraphics[width=.1\textwidth]{small_gvsu}
\end{figure}
\end{flushleft}
\restoregeometry
\newpage
% Define Page Geometry for rest of report
{\newgeometry{left=0.8in, right=0.8in, top=1in, bottom=1in}
% Page Numbers
\pagenumbering{arabic}
\pagestyle{fancy}
\fancyhf{}
\lhead{\ASSNMT}
\rhead{\leftmark}
\rfoot{Page \thepage}
% No paragraph indents
\setlength{\parindent}{0cm}
% =====----- Rest of Report -----=====
\newpage
\tableofcontents
\newpage
%%%%%%%%%%%%%%%%%%%%%%%%%%%%%%%%%%%%%%%%%%%%%%%%%%%%%%%%%% BACKGROUND %%%%%%%%%%%%%%%%%%%%%%%%%%%%%%%%%%%%%%%%%%%%%%%%%%%%%%%%%%
\section{Introduction}\label{introduction}
This document describes proposed changes to the Computer Engineering
undergraduate program at Grand Valley State University. These changes
are proposed based on suggestions drawn from the inferences and
experiences of a recent graduate of the program.

\bigskip

Specifically, this proposal will discuss a particular set of courses
related to the areas of \textbf{Programming Skills} and \textbf{Circuit
Design and Analysis}, which are the two pillars of Computer Engineering.

\bigskip

To that end, this proposal is intended to achieve four things:

\begin{enumerate}
\itemsep1pt\parskip0pt\parsep0pt
\item To give perspective on the usefulness and practicality of the existing curriculum
\item To critique certain aspects of the current curriculum that require review and careful attention
\item To emphasize the importance of certain topics that are currently \emph{not} part of the curriculum
\item To suggest a possible idealistic outline of a modified curriculum to better suit the current
environment of computer engineering, including ideas for new courses or projects
\end{enumerate}

Please note that this document conveys my personal opinion on the
different aspects of the curriculum, and while I believe the vast
majority of students, and even faculty, would agree on most of the
points, there may be ideas and topics discussed or criticisms presented
that do not align with the views of others or are lacking empirical
support in one way or another.

\bigskip

It also may be true that - as someone not involved in the social, political,
and financial institutions of the university - I may not be privy to certain
information or future plans, and what I suggest could be in the works or something that was discussed previously without my knowledge. Thus, suggestions on program additions/changes should be read as \emph{``Here is something that I believe could be improved upon.''}, rather than \emph{``This is what you should do, how you should do it, and when.''} I understand that there are vast political and financial complexities involved in departmental changes and course/curriculum modifications.

\bigskip

That being said, this document is really intended to give my perspective
as a former student in the program. I believe the engineering department
values the opinions of both former and current students, who have a
unique perspective on the content and impact of the courses in the
current curriculum. This particular perspective puts us in a spot to
give relevant and significant feedback on the program and the ways in
which it could be improved.

\bigskip

I have put a \textbf{significant} amount of personal time and thought into this document. I did this because I care about the program, and I believe that my feedback can help improve the curriculum. Please keep this in mind when reading this document.

\newpage
\section{Existing curriculum}\label{existing-curriculum}
As it stands, the following Electrical Engineering and Computer Science
courses are required for Computer Engineering graduates, and are the
courses focused on \textbf{Programming Skills} and \textbf{Circuit
Design and Analysis}:

\subsection{Engineering Courses:}\label{engineering-courses}
\begin{enumerate}
\itemsep1pt\parskip0pt\parsep0pt
\item EGR 261 - Structured Programming in C
\item EGR 214 - Circuit Analysis I
\item EGR 226 - Introduction to Digital Systems
\item EGR 280 - Probability and Signal Analysis
\item EGR 314 - Circuit Analysis II
\item EGR 315 - Electronic Circuits I
\item EGR 326 - Embedded System Design
\end{enumerate}

\subsection{Computer Science Courses:}\label{computer-science-courses}
\begin{enumerate}
\itemsep1pt\parskip0pt\parsep0pt
\item CIS 162 - Computer Science I (Java)
\item CIS 163 - Computer Science II (Java)
\item CIS 263 - Data Structures and Algorithms
\item CIS 350 - Introduction to Software Engineering
\item CIS 361 - System Programming
\item CIS 452 - Operating Systems Concepts
\end{enumerate}

\subsection{Computer Engineering
Electives:}\label{computer-engineering-electives}
\begin{enumerate}
\itemsep1pt\parskip0pt\parsep0pt
\item EGR 424 - Design of Microcontroller Applications
\item EGR 426 - Integrated Circuit System Design
\item CIS 457 - Data Communications
\item CIS 451 - Computer Architecture
\end{enumerate}

\newpage
\section{Benefits of Existing Curriculum}\label{benefits}
There are clearly benefits to the existing curriculum, and the vast
majority of the courses are well suited and in the proper order.

\bigskip

For example, the progression from EGR 214 to EGR 314
and EGR 315 is logical and smooth, and the latter half of the
CE curriculum, including CIS 452, CIS 457, and
EGR 424 is both challenging and incredibly insightful.

\subsection{Preparation and Practicality}\label{preparation-and-practicality}
In the end, the major benefit of Grand Valley's engineering program is
the raw practicality of the degree. I truly believe that I received a
better engineering education at GVSU than I would have at even larger
schools, including the University of Michigan. Our program is incredibly
strong when it comes to getting students in a position to contribute in
the workforce. While we may lag behind in research, our graduates feel
more comfortable using multimeters, breadboarding circuits, and creating
products from scratch. This puts us ahead of the curve in many ways, and
it is clear that employers are recognizing this with full force.

\subsection{Faculty Dedication and Knowledge}\label{faculty-dedication-and-knowledge}
The faculty is another strong aspect of the program. Rather than having
courses taught by graduate students, the professors are dedicated to
their classes and their students, and are extremely helpful both in and
out of class. Many of the faculty members here are very accomplished in
their field and they bring that knowledge and dedication to their
classes. Labs are often well defined by professors, and include
substantial practical training that supplements the course lectures.

\subsection{Cooperative Education Program}\label{cooperative-education-program}
Finally, one of the greatest benefits of GVSU's engineering program is
the Cooperative Education program. There is no doubt that our co-op
program is both unique and effective at preparing students to succeed
after graduation. More than just an opportunity for networking and a
steady paycheck, students learn valuable technical and social skills
during their co-op experience, and many discover the direction they do
(or do \emph{not}) want to take with their career once they graduate.
The magnitude of the benefit given by the co-op program is difficult to
measure, and is certainly one of the strongest areas of the entire
engineering program.

\subsection{Looking Further}\label{looking-further}
While there is much more I could say about the strengths of the great
program we have here at GVSU, the main focus of this document is
intended to present both a review of changes that could be made as well
as a supplimental look at what \textbf{other} things we can be doing to
prepare our students even more, and to ensure that the education they
receive, as it specifically relates to computer engineering, is as
comprehensive and useful as it can possibly be.

\newpage
\section{Downfalls of Existing Curriculum}\label{downfalls}

\subsection{Computer Engineering: A ``Frankensteining'' of Computer Science and Electrical Engineering}\label{frankenstein}
The field of computer engineering is historically a combination of
electrical engineering and computer science, and while the term
``Frankensteining'' may have a negative connotation to some, in some
ways it is a benefit; students participate in two related fields,
getting the best of both worlds. The CE program is an example of this,
in which we receive vast electrical engineering education as well as
important computer science education.

\bigskip

That being said, the computer engineering program has grown in recent
years, and it may be time to reevaluate the curriculum to more
appropriately address the needs of computer engineers directly, rather
than continuing to do so vicariously through the EE and CS departments.
Computer engineering is a vastly growing field; it has separated itself
from being simply a combination of electrical engineering and computer
science, and the computer engineering program at Grand Valley should
reflect that.

\bigskip

This is not to say that any \emph{major} program revisions are
necessary, as CEs should be taking the majority of their courses
in the CS and EE disciplines due to the nature of the degree, but rather that certain courses and course content which have been historically used at GVSU as part of the CE curriculum are not well suited for CE students, and perhaps specific CE courses should be created which are similar in the end goal of an existing course (such as teaching object oriented programming), but differ in their approach (such as doing so in C++ rather than Java), as will be described below.

\bigskip

\begin{quote}
\textbf{An aside:} I realize that GVSU is a publicly funded University, and in today's economy a department cannot simply hire new faculty at will to create new courses, but I would behoove the department to consider that software and embedded systems engineering (that is, Computer Engineering) is a field that is growing in an extraordinary manner, and thus you can anticipate that the number of students entering the program will only increase. I have absolutely no doubt that within a few years the number of Computer Engineering majors will equal or outnumber the Electrical Engineering majors.

\bigskip

Look back to the early 2000s; if a department or student had suggested that the university start investing more in teaching web frameworks, mobile applications, machine learning, or big data, and that they should create specific courses for those topics, the university would probably have scoffed at them. Yet, here we are in 2015, where those topics are \textbf{highly} desireable. What should we anticipate will happen with drone technology, wearable electronics, and the ``internet-of-things''? All of these fields use embedded electronics and signal processing. They all require \textbf{Computer Engineering}, not just Computer Science. Should we really fool ourselves into believing that the 32-bit microcontroller for the drone flight controller or the sensor chip on a smart watch will be coded in \emph{Java}? I don't think so.
\end{quote}

\subsection{Java I and II}\label{java}
Object Oriented Programming is without a doubt a fundamental aspect of
computer science and thus computer engineering. To graduate with a
degree in Computer Engineering and not have a basic understanding of OOP
is not a favorable achievement.

\bigskip

That being said, Java, the choice of language in our introductory OOP
courses (CIS 162 and 163), is not well suited for Computer Engineering
students. While it is a useful language, the domain for Java development
can be limited to specific areas and specific companies. For example,
Java development is highly concentrated in areas like Android
development, Enterprise systems, web development, and databases. All of
these areas are important, but for Computer Engineers, contrary to
Computer Scientists, these areas align less with both the rest of our
curriculum and potential job opportunities. Given that CEs do not take
any courses in mobile development, databases, or web development, the
use of Java after CIS 162 and CIS 163 for CEs is precisely nill, save
for a very limited number of local companies that may teach it during
co-op.

\bigskip

Given the time spent learning OOP, it occurs to me that instead of teaching
this concept in a language seldom used thereafter by CEs, a much more
logical approach would be to teach OOP using C++.

\bigskip

In addition to the benefits of C++ over Java as an educational language,
namely the advancement of the concepts of pointers, memory management,
and other important aspects which are `hidden' by Java, C++ actually
\emph{is} used in subsequent CE courses:

\begin{enumerate}
\item   CIS 263 attempts a very quick and limited introduction to C++, while
        the main focus of the course is Data Structures and Algorithms.
        Depending on the professor, students may have little experience
        actually \emph{coding} in C++ during the course, and many students
        complain about the lack of any real introduction to the language.
\item   C++ can greatly simplify projects in courses like CIS 457 and CIS 452,
        where access to data structures such as maps, trees, sets, and even
        simply class-based organization itself are less of a luxury and more
        of a necessity.
\item   Even on a microcontroller, C++ is useful. For example, I completed
        my EGR 326 project using C++ rather than C on an ATmega328P, and found
        this experience to be both technically rewarding as well as a more
        efficient use of my coding time.
\item   Looking back at all of the projects I completed in my upper-level CS courses, nearly all of them were in C++. A few of my project members and I were fortunate enough to have learned C++ either on our co-ops or on our own accord, but we had no formal training in C++, which could have served us well when it came to these senior courses. To be clear, these projects were not ``C++ projects''; rather, C++ was an option given to those students who happened to know it.
\end{enumerate}

Returning to the benefits of C++ as an educational tool for OOP, C++
offers more in the way of intellectual obligation than Java, and C++ is
arguably a more logical progression from C. Although both languages are
both based off C and yet very far from C in many ways, C++ shares, in
addition to many more, the following attributes with it's predecessor:

\begin{enumerate}
\item   The development environment for C++ is typically identical to that of
        C. With an install of \texttt{gcc}, the C++ compiler \texttt{g++} is
        already available and is nearly identical in use and function as the
        corresponding C compiler.
\item   Memory management, and specifically the existence of pointers is a key
        aspect of both C and C++, whereas Java shields the programmer from
        both the struggle and \emph{understanding} of pointers.
\item   The syntax of C++ is much closer to C than Java is to C. Take, as a
        very simple and admittedly contrived example, the following three
        declarations:

\begin{lstlisting}[language=C,label=c-code]
    char array[256];    //C
\end{lstlisting}

\begin{lstlisting}[language=C++,label=c-code]
    char array[256];    //C++
\end{lstlisting}

\begin{lstlisting}[language=Java,label=c-code]
    char[] array = new char[256];   //Java
\end{lstlisting}

Clearly, in this simple scenario, Java's syntax is very much different
than C, and can even be confusing (i.e. where to place the
\texttt{{[} {]}}), which brings us to the next point:

\item   C++ supports inline C. This means that when developing in C++, you
        could write your entire program in C and compile it with a C++
        compiler, and the program will function the same as if you had done it
        with \texttt{gcc}. In fact, in the previous example, the array
        declaration \textbf{was} just plain C.
\end{enumerate}

These similarities also mean that there could even be \emph{one} OOP course in C++ rather than two in Java. Most students will agree that CIS 162 is extremely easy and that any serious programmer could very easily begin at CIS 163 (Java II) without much difficulty. Using C++ will only further relax the need for two courses, since it is such a natural progression from C. If CE students take the C++ course after taking any of the C courses (EGR 261, CIS 361, or EGR 226), I believe that only one OOP course would need to be required. This frees up one required course, which could be replaced by EGR 424 or CIS 457, as will be described later.

\bigskip

\begin{quote}
\textbf{Edit:} I understand that a new 1-credit course, CIS 159: \emph{Intro to Java for C Programmers}, will be introduced as a replacement for CIS 162. I think this is a great step in the right direction, as I think everyone would agree that CIS 162 was unnecessary. However, the end goal for the CE department should be geting rid of both CIS 159 \emph{and} CIS 163 for their students, and instead teach C and C++ sequentially.
\end{quote}

\bigskip

C++ is also more widely used than Java in the field of computer
\textbf{engineering}. I will not attempt an argument at programming
language popularity in general (see statistics at
(\url{http://www.tiobe.com/index.php/content/paperinfo/tpci/index.html}),
but clearly Java's use in Android development is a vast porportion of
it's popularity, and it is clear that C and C++ are the most widely used
languages in computer engineering. In my personal experience, through
all three of my co-ops I used C once on a piece of avionics software,
and C++ on both of the other two: once on a sub-orbital balloon
satellite at NASA and again on a large particle physics project at CERN.
At both NASA and CERN, C++ was by far the most prevalent language used
by the engineers.

\bigskip

In the end, all of these attributes make C++ a more logical progression for the foundational OOP Computer Engineering course(s) than using Java.

\subsection{System Programming}\label{system-programming}
CIS 361 is split into two distinct halves:

\begin{enumerate}
\itemsep1pt\parskip0pt\parsep0pt
\item System programming in C in a Linux environment
\item Shell programming in a Linux environment
\end{enumerate}

Both of these topics are extremely important and must be part of the CE
curriculum. The CE Committee did a great job of replacing EGR 209 (Statics) with CIS 361, and I commend them for that.

\bigskip

However, the problem lies in the way the first half is typically
instructed (at no fault to the instructor); since CIS 361 is offered as
an elective for CS students (it is required for CEs), the first half of
the course inevitably begins as a \textbf{``Java to C''} course, rather
than a course on system programming. In reality, the course \emph{should} be offered to both majors, but changes need to be made.

\bigskip

Since CS students have little to no introduction to C, they often
struggle, and thus the instructor must start with a \textbf{very} basic
introduction to C: how to declare variables, how arrays work, what a
string is in C, and an overview of pointers, which ultimately baffles
many CS students who have never encountered the topic before. Thus, for
two weeks the CS students struggle to catch up, while the CE students do
their best to remain attentive while they review the most basic of C
topics. Clearly there exists a better solution.

\bigskip

Perhaps the answer is to stop giving an introduction to basic C syntax to CIS students, and allow them to learn on their own through experimentation during the first few weeks of the course. This would not be dissimilar to what both CIS and CE students must do when they encounter CIS 263, which is often instructed in C++ with no primer to the language. Then again, perhaps we should just drop the basic C syntax lessons and assume CIS students will figure out the peculiarities, but still give a review of pointers. In the end, the course is on system programming, not C syntax, and halfway through the semester we drop all C programming completely and switch to shell scripting.

\bigskip

To be sure, the latter half of the course is both useful and engaging
for both CS and CE students, but more on scripting will be discussed later in the document in Section \ref{extracurricular}.

\subsection{Hardware Design Skills, or, ``Arduino Dependence''}\label{arduino-dependence}
EGR 214, 226, 314, 315, and 316 offer a broad range of both analog and
digital circuit design and analysis concepts. Since CEs are often more
concerned with the digital side of circuit design, so too will be this
suggested improvement to the existing courses.

\bigskip

Out of the 30+ EE and CE students (a number which increases annually), by
the senior year, few if any have real, practical PCB design skills. To
be sure, PCB design topics are \emph{discussed} in EGR 326, and some
courses even require designing small boards or `shields', but the
reliance of students on the \textbf{Arduino} platform mitigates any
opportunity for real practical experience when it comes to learning the
ins and outs of PCB design. Mention the words `gerber files', and many
EE and CE students will cringe, or even worse, return a blank stare.

\bigskip

Additionally, the lack of preparation for students when it comes to
board design leads professors to be reluctant towards surface mount
components and non-breadboardable chips. These types of packages are not
only the most common in any real circuit design, but surface mount
soldering techniques, including the use of solder paste, are a basic
element of being an electrical or computer engineer.

\bigskip

While we are not totally missing the boat on this, since it has been pointed out that many programs also do not offer much in the way of PCB design, I see no reason to follow in the footsteps of other schools that are lacking in content. GVSU should be proud of how it prepares students for the workforce, and PCB design is an important part of this. Sure, few if any students will graduate with full-time jobs as dedicated PCB designers, but that is missing the point: PCB design gives students more appreciation for where their boards come from and how they work on an intimate level. Debugging an Arduino is much easier if you designed it and printed it yourself. And when our engineering graduates \emph{do} get their first job, they will appreciate the guys in the PCB design team that much more. Moreover, it is just not that difficult. Designing your own Arduino is really quite trivial, given that the ATmega328P microcontroller has nearly everything contained inside.

\bigskip

I feel there is a significant opportunity for students to improve their
PCB design skills by foregoing the Arduino platform and having students
design their \textbf{own} development board, based around the ATmega328P
(the same microcontroller found on the Arduino), sometime during their
sophomore or junior year.

\bigskip

For example, during EGR 214 or EGR 226, students could design the
schematic and layout their own ATmega328P-based development board to use
in EGR 226 and other courses (EGR 326). In all truth, designing a
development board around the ATmega328P is very simple. To get a basic
functional environment, the following components are necessary:

\begin{enumerate}
\itemsep1pt\parskip0pt\parsep0pt
\item An ATmega328P
\item A couple of capacitors
\item A couple of resistors
\item A few LEDs
\item A 5V regulator (Like an LM7805)
\item A reset button
\item A fuse
\item A barrel connector
\item A six-pin ICSP header for programming the chip
\end{enumerate}

And optionally:

\begin{enumerate}
\itemsep1pt\parskip0pt\parsep0pt
\item An external oscillator (with two caps)
\item A USB connector
\item An FTDI chip to program it over USB instead of ICSP
\item Tx and Rx LEDs
\end{enumerate}

This could even be breadboarded with minimal complications as a pre-lab,
and then converted into a schematic and ultimately manufactured. To save
time, instructors could prepare their own dev board schematics and have
many extra PCBs printed, both so students can reference their designs as
well as in case students are unable to complete the board design
themselves, so they are prepared for future courses which will use the
board (of course, you don't have to tell the students that there is a
plan B if they don't finish\ldots{}).

\bigskip

In addition to hardware experience, this would introduce students to
open source software such as \texttt{avrdude} and \texttt{avr-gcc},
rather than allowing them to use the Arduino IDE and it's vast libraries
which, although helpful, do not facilitate learning how things work
`behind the scenes'. Students would relinquish their reliance on the
Arduino IDE and the Arduino library environment, and instead it would
force them to code in a traditional text-editor environment, use command
line tools like makefiles and debuggers, program their microcontroller
using the ICSP interface and \texttt{avrdude} directly, and communicate
with the board using open source command line UART programs like
\texttt{minicom}, all of which are incredibly valuable skills to have.

\bigskip

A logical next step for this would be when EEs and CEs take EGR 326. In
EGR 326 we learn about regulator design, how to protect digital I/O, and
in general how to make your embedded design more robust. What better
scenario could there be to make a \textbf{rev 2} of the student's
development board, in which they re-design the regulator (maybe use two
LM317s in parallel with ballast resistors or something), implement
safety features like using zener diodes and resettable fuses on digital
pins, and learn the importance of flyback diodes and other useful
components. I believe that this step in the development process, of adding in safety and reliability features to their own development board, is the most important of all, and really should be considered for future courses.

\bigskip

EGR 326 does scratch the surface of \emph{theoretical} board layout,
spotlighting the importance of, for example, putting your filtering caps
as close as you can to certain pins, or ensuring matched
impedence on high-speed transmission lines. This should all be
practically expressed during the design of the \textbf{rev 2} board,
wherein students apply these techniques to build a much more robust
development board; think `Ruggeduino'. This would be the main project
throughout EGR 326, culminating in the manufacturing of their board,
which could be used in a very \textbf{simple} project at the end of the
semester. As it were, the EGR 326 project we completed was heavily
comprised of topics that were not the main focus of EGR 326, such as
infrared diodes and mechanical design including heavy reliance on
3D printing. Although these things are important, the design of a robust
development board aligns more closely with the content of EGR 326, and
in the end is something that students can be proud of and keep using on
different projects long after they have graduated.

\begin{quote}
\textbf{Side note}: Perhaps it makes sense to use an Arduino for EGR 226
and then graduate to their own development board in EGR 326, but in any
case it is extremely important to have a practical experience with board
design and digital systems.
\end{quote}

These finished boards do much more than simply serve as a development
board for them in future courses; the board will be a source of pride
and individuality for students, who may even become excited about board
design and find that they quite enjoy it.

\bigskip

\textbf{Even more paramount is the ability for students to showcase
their board during interviews.} They will have at least one board
(and possibly a second revision) to physically hand to prospective employers,
while they explain the steps they took to design and enhance the board,
and could even show the documentation they have prepared surrounding the
board, such as schematics and calculations. This gives students an
opportunity to empirically show that they can design a printed circuit
board, and indeed a complete embedded system, to a specification and
that they have improved upon it with documentation and critical
thinking.

\bigskip

In addition to the hardware aspect, there exists an opportunity for
students to interact with full-featured, open source software, and for
partnerships to be made with alternative board manufacturers.

\bigskip

The fully open-source schematic and PCB design tool KiCad
\href{https://www.kicad-pcb.org}{(www.kicad-pcb.org)} is produced by a
collaboration at CERN (the European Center for Nuclear Research), and is
constantly undergoing bug fixes and feature enhancements. The software
is fully cross-platform, with builds for OS X, Windows, and Linux. There
are nightly builds and ongoing development on the tool, which is used
around the globe due to it's 100\% free open source commitment; widely
available documentation and support community; and features like
push-and-shove routing, which are traditionally found on only extremely
expensive industry tools.

\begin{quote}
\textbf{Side note}: KiCad is due for an official `stable' release within
the next couple of months. The current builds are `stable' in the sense
that they function well, but an official stable build is coming soon.
\end{quote}

OSH Park \href{https://www.oshpark.com}{(www.oshpark.com)} is a community PCB
manufacturing platform. They are based in the US (in Oregon) and offer
both 2-layer and 4-layer boards for incredible prices. They charge just
\textbf{\$5 per square inch} for 2-layer boards, and they are guaranteed
to arrive within two weeks. In addition, the price always includes
\textbf{three} copies of the board, not one. We have used OSH Park for
many of our projects and have had very good service, including being
upgraded to a much faster turnaround time and shipping due to being
repeat customers.

\bigskip

Their community model is based on the premise that they will fill up an
entire panel with boards from different orders, and then print and ship
the boards when a panel is full. Thus, there exists an opportunity for
the SoE to collaborate with OSH Park for printing the boards of all
students in any given course, who's boards would likely fill up a
significant portion of a panel, increasing the turnaround time for board
production.

\bigskip

Regardless of whether KiCad or Eagle is chosen for the design tool, and
regardless of the board manufacturer, the ability for both EE and CE
students to design schematics and physical boards is a
\textbf{fundamental area of knowledge and skills}, and separates purely
theoretically-focused engineers from those who have applicable skills
desired by industry, research, and major technology companies alike.

\subsection{Computer Engineering Electives}\label{electives}
The current offering for CE electives basically boils down to the
following options:

\begin{enumerate}
\itemsep1pt\parskip0pt\parsep0pt
\item EGR 424 - Design of Microcontroller Applications
\item EGR 426 - Integrated Circuit System Design
\item CIS 457 - Data Communications
\item CIS 451 - Computer Architecture
\item One or two EE or biomedical courses
\end{enumerate}

Of the four CS/CE electives, I would argue that two are in fact
\textbf{fundamental} topics, and should be \textbf{required} by all CE
students. These two courses are \textbf{EGR 424} and \textbf{CIS 457}.

\bigskip

\textbf{EGR 424} is a very fundamental look at how a microcontroller
functions and the intricate relationship between the code, the compiler,
the linker, and the target architecture. \textbf{If any class is truly
`computer engineering', it is EGR 424.} In addition to being arguably
the most fundamental of all CE courses, it is also a wonderful companion
to courses like CIS 452, which \emph{is} a required course. Having taken
EGR 424 before CIS 452, there were a generous helping of topics which
were made absolutely more clear by having actually implemented a
multi-threaded kernel in C.

\bigskip

\textbf{CIS 457} introduces the concepts of packet-switched networking;
network programming (sockets, etc.); the internet and TCP/IP; and a
basic introduction to network security. These concepts are also
fundamental to computer engineering, and are without a doubt the most
well connected to the growing presence of internet-connected devices and
security concerns today. The field of computer engineering has been
moving towards connected devices and the `internet of things' for quite
some time, and a basic understanding of computer networking is the
foundation of this movement.

\bigskip

To receive a degree in Computer Engineering without EGR 424 and CIS 457
is, in my opinion, a huge setback, and the consequence is a lack of
preparation necessary to succeed in the field.

\bigskip

There are also some EE electives, namely the new \textbf{Embedded
Systems Interface} course by Dr.~Bossemeyer and Dr.~Jiao that would be
of great value to CEs as well as EEs.

\bigskip

\begin{quote}
\textbf{Edit:} I have been informed that the Embedded I/F course is already an approved CE elective. Great work!
\end{quote}

\subsection{Software Engineering}\label{software-engineering}
CIS 350 is intended to be a course in software engineering, which
demonstrates the lifecycle of software projects and different project
management schemes. This is an important area of computer engineering,
and is a necessary part of the curriculum. However, in practicality, CIS
350 fails to prepare students for any meaningful concept of software
engineering and the project management process. Due mainly to a lack of
any practical examples or hands-on learning, CIS 350 is merely an
apathetic theoretical discussion of Scrum and Agile development with a
forced SWS component. The end result is that students are left wondering
what, if anything, they received from the course, and why it is even a
course at all.

\bigskip

A major improvement to CIS 350 would be to begin the course with a fake
project proposal/set of specifications by a fake company, and to follow
the development of that software project throughout the semester with
practical examples as the different topics are introduced and discussed.
The entire course could be engaged in demonstrating the benefits of
certain aspects of software engineering in a \textbf{practical} manner,
as it applies to a specific project example.

\bigskip

A theoretical look at software engineering is not only boring and not
intellectually stimulating, but due to the number of different tools and
methods available, the instructor can, at best, give a very limited view
of any given methodology. Thus, I would argue that it would better suit
the students to take a much broader view of software engineering and
discuss only a select number of topics in greater detail. To complement
this, real examples of UML charts, state machine diagrams, etc. would be
created by the professor and students in concert along the way, and the
instructor could address the different ways of surmounting a large
software project with real, physical examples.

\bigskip

Finally, CIS 350 could even have room for an introduction to version control systems, a topic that I \textbf{really} think we are missing out on at GVSU. I will discuss much more about version control next, in Section \ref{git}.

\newpage
\section{Extracurricular Topics}\label{extracurricular}
There are a vast number of topics that \emph{could} be covered in a CE
curriculum, but due to staffing and financial constraints, it is
understandable that only a small subset of these extracurricular topics
can be implemented into a program (probably as electives).

\bigskip

However, the following topics, not necessarily courses of their own, are
highly valuable and are in growing demand for today's computer
engineering graduates:

\begin{enumerate}
\itemsep1pt\parskip0pt\parsep0pt
\item Computer security
\item Connected devices
\item Scripting
\item Linux development
\item Version control systems
\end{enumerate}

\subsection{Computer Security}\label{computer-security}
Computer security is a field that is growing immensely, and there should
be a focus at GVSU to take advantage of an emerging field.

\bigskip

For CS students, there is an option to take CIS 458, which gives at
least an overview of security and modern cryptography concepts. In an
ideal world, there would instead be two courses: Network and System
Security, and Cryptography. Both of these courses would be offered to CS
and CE students.

\bigskip

Additionally, for CE students there could also exist an FPGA
Cryptography course, since FPGAs are often used in encryption/decryption
schemes due to their high-performance computing abilities. I believe
Dr.~Parikh has done research in this area.

\bigskip

My job after graduation will be in the field of computer security. While
I did gain some knowledge from CIS 457 (Data Communications), which
briefly discusses it, I learned quite a bit of what I know from
open-source online courses from other universities, and almost all of my
knowledge will be gained on the job. This is an example of where (a)
course(s) in computer security could prove to benefit many students in
their job search, giving them a definite advantage over other students
who lack such knowledge.

\bigskip

I know many students are very interested in the field, as it is one of
the more provocative areas of computer science. There is no doubt that
CIS 458 fills up quickly whenever it is offered by Dr.~Kalafut.

\bigskip

\begin{quote}
\textbf{Edit:} It was brought to my attention that there was previously a proposal for students to emphasize in either security or databases, and I believe this is a great idea.
\end{quote}

\subsection{Connected Devices}\label{connected-devices}
\begin{quote}
\textbf{Edit}: EGR 436 - Embedded Systems Interface already
exists and covers many of these points, and I was not aware that it is now offered as a CE elective.
\end{quote}

The so-called `Internet-of-Things' is also an emerging field, and one
that is easily intermingled with existing topics, such as embedded
system design. Such a course in connected devices could build on the
advanced embedded system design class, and projects could use bluetooth
or WiFi connectivity by way of designing small boards to interface with
an existing development board, or using existing development boards that
include wireless interfaces (both Atmel and TI have many).

\subsection{Scripting}\label{scripting}
Scripting is an important aspect of computer science and engineering,
and is a basic skill that should be held by anyone graduating with such
a degree. For example, Python is a highly useful language with
applications ranging from web design to data processing. Python is
incredibly easy to learn, and is very powerful for certain areas of
application. Perhaps this would not be suited for it's own course, but
Python scripting could be included in courses like CIS 361.

\subsection{Linux Development}\label{linux-development}
There is a focus on Linux development in CIS 361, CIS 452, and CIS 457.
However, there should be a focus on open-source Linux development in
many more areas of the CE curriculum. The benefits of open source tools
and software are becoming more clear as time goes on, and a computer
engineer without a solid knowledge of Linux is not a computer engineer
that is highly desired by many companies. An emphasis on using open
source tools and an emphasis on command line development is essential
for the appreciation of large IDEs that take much of the education out
of programming. While IDEs are very useful and will surely be used once
engineers graduate and obtain jobs, the whole point of an IDE is that it
can be easily learned and companies don't expect prospective employees
to be fluent in their exact IDE or toolchain environment; what they do
expect is that engineers are knowledgeable in the fundamentals of the
coding process and environment, and developing using traditional text
editors and command line tools is the only way to obtain such knowledge.

\bigskip

An additional enhancement in terms of Linux development would be to
create a \texttt{gvsu-egr} package repository that could be installed
easily using any Linux package manager: on Ubuntu and Debian systems
using \texttt{apt-get install gvsu-egr} or other systems with tools like
\texttt{yum} or \texttt{zypper}. This would allow professors to add
software required for their class to the central repository and then
students could easily install and upgrade this software without wasting
time on a complex installation. While there is educational merit in
learning to install tools on a Linux system, it is likely true that
student's time is better spent learning to use the tools rather than
debugging their faulty installation.

\subsection{Version Control Systems}\label{git}
The education and use of version control systems, like \texttt{git}, in
my opinion, \textbf{may be the most important of all the points made in
this entire proposal}. Version control systems are so foundational to
software development and software engineering that they are used by
nearly all creditable companies worldwide.

\bigskip

Not only that, but version control offers incalculable benefits to
\textbf{students}:

\begin{enumerate}
\itemsep1pt\parskip0pt\parsep0pt
\item   Students using version control will not mysteriously lose their
        projects; it is less chaos for both students \emph{and} professors.
\item   Students using version control are able to more quickly recover from
        mistakes made during the development process by rolling back changes
        to their code or taking advantage of branching schemes.
\item   Version control systems make \textbf{team} development
        \emph{possible}; After using \texttt{git} for team projects, it is
        extremely difficult to imagine doing so without it.
\item   The popular \texttt{git} repository hosting website, \texttt{GitHub}
        \href{https://www.github.com}{(www.github.com)} is now viewed by many
        companies not as a \emph{suppliment} to a traditional resume, but
        rather as a nearly-obligatory resource through which they can view a
        candidate's work and really see if they produce quality code.
\end{enumerate}

All of these benefits of version control systems, and more specifically
of \texttt{git}, one of the most popular and easiest to use, make it an
absolutely \textbf{essential} aspect of a modern computer science or
computer engineering degree. I truly cannot imagine working on a team or
any sort of coding project now without \texttt{git}, and my
\texttt{GitHub} account link is the only other hyperlink on my resume
other than my own personal website. \texttt{git} and \texttt{GitHub},
along with any version control system, are so prolific and unavoidable
in today's software world that they simply \textbf{must} be a part of
any engineering student's education.

\bigskip

Version control systems are a perfect fit for courses like CIS 350 -
Software Engineering, if taught properly. The only trouble with
\texttt{git} (and any version control software) is that if learned
incorrectly, or not learned thoroughly enough, students can make
mistakes that affect their entire project team. With version control,
however, these mistakes are easily recovered from, and the entire
history of their project is backed up in well-defined stages.

\bigskip

\texttt{git} is easy to learn and has a \textbf{ton} of wonderful
documentation behind it (check out \url{https://help.github.com/}).
\href{https://try.github.io/levels/1/challenges/1}{This} interactive
website (\url{https://try.github.io/levels/1/challenges/1}) is a
fantastic example of how easy it is to get started using \texttt{git}.

\bigskip

It is not only for the benefit of students that they use version control
systems; professors will also benefit greatly. Instead of a chaotic mess
of emails with 40 files named \texttt{helloworld.c} in their downloads
folder, professors can simply receive a link to a student's repository.
Even submitting their code can become an easy process. If students take
advantage of \texttt{git} and \texttt{GitHub}'s very simple release
functionality, submitting an assignment can be as simple as running
\texttt{git tag -a \"gibson-v1.0\"}, which automatically creates a zip
folder of the student's project that they can download and submit. For
group projects, professors can even see detailed statistics and graphs
of who is contributing what to any given project, if they so choose to
do so. To avoid plagiarism between students in classes, projects could
be hosted on other sites such as Gitlab (\href{https://www.gitlab.com}{www.gitlab.com}) or Bitbucket
(\href{https://www.bitbucket.org}{www.bitbucket.org}), both of which offer free private repositories.

\bigskip

I propose that instructors in CS and CE courses actively promote using
\texttt{git} and using version control schemes such as
\href{http://nvie.com/posts/a-successful-git-branching-model/}{this}
incredibly powerful and easy to use branching model.
(\url{http://nvie.com/posts/a-successful-git-branching-model/}). In the
end, using \texttt{git} will benefit both students and instructors and
will result in every student who graduates from GVSU having their own
personal portfolio of the code they have written readily available to
show off to their potential employers.

\begin{quote}
\textbf{Side note}: Everyone has their own favorite version control
system, and while \texttt{git} is certainly one of the more popular
options, others great options do exist. However, to keep things
consistent, a single option should be chosen. \texttt{git} is a perfect
choice for this scenario, because it is open source, distributed, very
easy to learn, has a mountain of great documentation behind it, and is
exclusively used by the most popular public repository site, GitHub.
\end{quote}

% Curriculum section
\section{Curriculum Suggestions}\label{curriculum-suggestions}
While I am aware that changes to a curriculum are not so easily implemented that the school can simply add or remove courses at will, the curriculum suggestions below are my view of what I believe to be a generally improved curriculum for Computer Engineering students. These suggestions stem from my experience with the existing curriculum, the needs of co-op employers, and the insight young engineering students have in new topics and fields, having grown up in the era of massive technological changes in the field of computer engineering.

\bigskip

Again, this is essentially just a very \textbf{idealistic} view of what a curriculum might look like, so it could be referenced if changes were to be made.

\bigskip

Note that the only courses included are those discussed in this proposal, and not any gen-ed courses or other pre-requisite engineering courses.

\subsection{Freshman Year}\label{freshman-year}
\textbf{Foundational Courses}

\subsection{Sophomore Year}\label{sophomore-year}
\paragraph{Fall}\label{sophomore-fall}
\begin{enumerate}
    \item   Circuit Analysis I
\end{enumerate}
\paragraph{Winter}\label{sophomore-winter}
\begin{enumerate}
    \item   Introduction to C Programming
\end{enumerate}

\subsection{Junior Year}\label{junior-year}
\paragraph{Fall}\label{junior-fall}
\begin{enumerate}
    \item   System Programming
    \item   Introduction to Digital Systems
\end{enumerate}
\paragraph{Winter}\label{junior-winter}
\begin{enumerate}
    \item   Introduction to C++ Programming
    \item   Data Structures and Algorithms
    \item   CE Elective
\end{enumerate}
\paragraph{Summer}\label{junior-summer}
\textbf{Co-op}

\subsection{Senior Year 1}\label{senior-year-1}
\paragraph{Fall}\label{senior-fall-1}
\begin{enumerate}
    \item   Circuit Analysis II
    \item   Electrical Circuits I
    \item   Advanced Digital Systems
\end{enumerate}
\paragraph{Winter}\label{senior-winter-1}
\textbf{Co-op}
\paragraph{Summer}\label{senior-summer-1}
\begin{enumerate}
    \item   Software Engineering
    \item   Design of Microcontroller Applications
    \item   CE Elective
\end{enumerate}

\subsection{Senior Year 2}\label{senior-year-2}
\paragraph{Fall}\label{senior-fall-2}
\textbf{Co-op}
\paragraph{Winter}\label{senior-winter-2}
\begin{enumerate}
    \item   Operating Systems Concepts
    \item   Data Communications
    \item   CE Elective
    \item   Senior Project I
\end{enumerate}
\paragraph{Summer}\label{senior-summer-2}
\begin{enumerate}
    \item   Senior Project II
\end{enumerate}

\subsection{Electives}\label{suggested-electives}
\begin{enumerate}
    \item   EGR 426 - Integrated Circuit System Design
    \item   CIS 458 - System Security
    \item   EGR 436 - Embedded Systems Interface
    \item   CIS 375 - Wireless Networking Systems
    \item   CIS 461 - Compiler Design and Construction
    \item   CIS 451 - Computer Architecture (Although this is really covered quite well between EGR 226, EGR 424, and CIS 452)
\end{enumerate}

\section{Thank You and Request for Feedback}\label{request-for-feedback}
Thank you very much for reading this document. By considering the feedback from former students, the engineering program will offer even more to its future students, and a reputation for \emph{actually} considering feedback goes a long way.

\bigskip

Also thank you to the CE department and CE committee. The changes you have made thus far have brought the CE department a very long way, and I believe some of the points I have mentioned are already being discussed. However, there is still room for improvement.

\bigskip

I have put a considerable amount of time into this document, and would very much appreciate \textbf{your} feedback. Just as feedback from students regarding courses can be useful, feedback from faculty regarding student ideas is essential.

\bigskip

\textbf{Please send feedback regarding this document to: Joe Gibson  \href{mailto:gibsjose@mail.gvsu.edu}{(gibsjose@mail.gvsu.edu)}.}

\bigskip

You can also view the source code for this document (written in LaTeX) on my GitHub page at \href{https://www.github.com/gibsjose/CE-Proposal}{www.github.com/gibsjose/CE-Proposal}.

\end{document}

% ======== For Reference =============
% H parameter for the figure environment
% keeps it from floating
\begin{figure}[H]
	\centering
	\includegraphics[width=0.8\textwidth]{FIG2_1b}
	\caption{Convolution of Identical Unit Step Sequences}
	\label{fig:21b}
\end{figure}
% To turn off caption numbering place this:
%\captionsetup[figure]{labelformat=empty}
% Before the figure, To turn back on:
%captionsetup[figure]{labelformat=default}
\begin{equation}
	\begin{array}{rcl}

	y[n] &=& 0.5x[n]+x[n-1]+2x[n-2]\\
	y[n] &=& 0.8y[n-1]+2x[n]\\
	y[n]-0.8y[n-1] &=& 2x[n-1]
	\end{array}
\end{equation}

\begin{subequations}
	\begin{align}
		h[n] &= 2\delta[n+1]-2\delta[n-1]\\
		x[n] &= \delta[n] + \delta[n-2]
	\end{align}
\end{subequations}

% Resize used to make table width of text, may be omitted
\begin{table}[h]
\resizebox{\textwidth}{!}{
\centering
\begin{tabular}{|c|c|c|c|}
\hline
Case & $Z_c$          & $l$           & $c$       \\ \hline
1    & $110.9 \Omega$ & 0.593 $\mu H$ & 0.048 nF  \\ \hline
2    & $171.3 \Omega$ & 0.803 $\mu H$  & 27.408 pF \\ \hline
3    & $327.3 \Omega$ & 1.102 $\mu H$  & 10.294 pF \\ \hline
\end{tabular}}
\caption{Calculated Parameters}
\label{tbl:calcd}
\end{table}

% Code Snippet:
\begin{lstlisting}[language=C,label=lala,caption=this thing]
  code snippet
\end{lstlisting}

% bulleted list
\begin{itemize}
\item this is an item
\end{itemize}

\renewcommand*\contentsname{ }
\tableofcontents
\listoffigures
